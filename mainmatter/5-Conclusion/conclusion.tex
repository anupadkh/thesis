\chapter{Conclusions}

The research was conducted for series of experiments using different CNN architectures and validated using four different settings of drug and protein combinations. The experiments show that the network requires past knowledge on proteins and drugs to yield the best results. The model could yield best scores when both the protein and drug were present in the training phase. We also conclude that if the domain has fewer parameters to learn from, we can represent them in new dimensions to achieve better results in machine learning problem.

Finally, it concludes that:
\begin{itemize}
    \item Proteins need to be represented in higher dimensions of R2RSRV factors and the domain representation by using PSSM features along with the sequence information for formulating protein-solving network architecture.
    \item Same filter size of 8 can be used to train the CNN layer for both proteins and drugs.
\end{itemize}

\chapter{Recommendations}
% \section{Future Works}

At present, the sample space of~\acrfull{nfl} has been explored by forming the different feature-sets. The different feature sets on chemical values of drugs can be used to represent a much broader spectrum of drug molecule representation. In case of protein, protein folds could be the incorporated information which could be supplemented for the available 3D protein structures and a good prediction algorithm chosen for CASP\citep{CASP82008}. For the features being evaluated, the Grid-Search CV could be applied over various algorithms and Stacking Generalization could be used to create a better optimized machine learning solution to predict protein-drug interaction score from sequence information. Successively, important is a correct modeling of Pharmacophore modeling of drug-proteins pairs so that it can be applied directly to medical supervisions. 

