% This file doesn't compile!!! Its only purpose is to help the person who is writing the document.

% Figure

\begin{figure}[tb]
\centering
\includegraphics[width=0.3\linewidth]{introduction/fig/Felix_the_cat.pdf}
\caption{Felix the Cat}
\label{fig:felix}
\end{figure}


% subfloat

\begin{figure}[tb]
	\centering
	\subfloat[A horrible one.]{
		\label{fig:fig2sub1}
        \includegraphics[width=0.5\linewidth]{introduction/fig/figure1.jpg}}
	\subfloat[A proper one.]{
		\label{fig:fig2sub2}
		\includegraphics[width=0.6\linewidth]{introduction/fig/figure2.pdf}}
	\caption{A figure with two subfloats.}
	\label{fig:fig2}
\end{figure}


% Landspace figure

\begin{landscape}
	\begin{figure}[H]
\centering
\includegraphics[width=0.5\linewidth]{introduction/fig/Felix_the_cat.pdf}
\caption{Here's a large drawing of Felix the Cat that wouldn't fit in a portrait page}
\label{fig:felix2}
\end{figure}
\end{landscape}


% Footnote

Write the text and put a footnote this way \footnote{Note explanation}

% Citation

Use the function \cite{gowar1989power}, where \emph{gowar1989power} is the reference inserted in the sample.bib file.

% Equation

Use the website : https://www.codecogs.com/latex/eqneditor.php

\begin{equation}
v(x)=\frac{1}{2}\sin(2 \omega t + \phi) e^{-j s t}
\label{eq:cacona}
\end{equation}

\begin{samepage}
Sometimes, the symbols in an equation are defined as follows\footnote{Some authors like to define their symbols this way.}:
\begin{equation}
	V(t)=A \sin(\omega t+\theta_0)
\end{equation}
\begin{tabular}{lll}
	where & $V$ & is a voltage waveform,\\
	& $A$ & is the amplitude of the voltage,\\
	& $\omega$ & is the angular frequency,\\
	& $t$ & is the time.
\end{tabular}
\end{samepage}

% Tables

Use the website : https://www.tablesgenerator.com/

\begin{table}[tb]
	\centering
	\caption{Characteristic parameters of the system}
	\begin{tabular}{lll}
		Parameter & Value & Units\\
		\hline
		$P$ & 1 & kW \\
		$Q$ & 0 & kVAr\\
	    \hline
	\end{tabular}
	\caption{Characteristic parameters of the system}
	\label{tab:tab1}
\end{table}


% Algorithms and code

Insert the package \usepackage{listings} and search for the configuration of the language you want to use. For example : 

\usepackage{listings} % To use source code

\begin{lstlisting}[caption=Source code for {\it hello.m},label=lst:code1,breaklines=true,basewidth=4pt,prebreak=**,postbreak=**,frame=single]
for i:=maxint to 0 do
begin
{ do nothing }
end;
Write('Case insensitive ');
Write('Pascal keywords.');
\end{lstlisting}
   

% Titles page

Doctorate Thesis : 

https://www.overleaf.com/learn/latex/How_to_Write_a_Thesis_in_LaTeX_(Part_5):_Customising_Your_Title_Page_and_Abstract

Rapport Français de Stage

https://www.overleaf.com/latex/templates/graduation-project-cover-page/qtmbhjbwjqcd

% Acronyms
\acrshort{lcm}
\acrlong{lcm}
\acrfull{lcm}
